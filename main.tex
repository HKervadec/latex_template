\documentclass[11pt, a4paper, twocolumn]{article}
\usepackage[english]{babel} % Explicit language
\usepackage[T1]{fontenc} % Encoding
\usepackage[utf8]{inputenc} % Encoding
\usepackage{lmodern} % Font
\usepackage{graphicx} % To insert figures
\usepackage{hyperref} % Clickable links
\hypersetup{
    colorlinks,
    citecolor=black,
    filecolor=black,
    linkcolor=black,
    urlcolor=black
}
\usepackage[left=2cm, right=2cm, bottom=3cm, top=3cm]{geometry} % Borders

\title{Minimal template for Science}
\author{Hoel \textsc{Kervadec}}
\date{\today}

\begin{document}
    \maketitle

    \begin{abstract}
        Ce n'est qu'en essayant continuellement que l'on finit par réussir....
        En d'autres termes... Plus ça rate et plus on a de chances que ça marche...
    \end{abstract}

    \setcounter{tocdepth}{2}
    \tableofcontents

    \setlength{\parskip}{5pt}
    \chapter{Introduction}
    This small template is aimed at making my life easier in the future.

    I will add with time a list of useful packages and what they do, to avoid too much bloat in new projects.

    \section{Conclusion}
    This was really useful.

    \bibliographystyle{plain}
    \bibliography{input/biblio}
\end{document}