\documentclass[10pt, letterpaper]{book}
\usepackage[english]{babel} % Explicit language
\usepackage[T1]{fontenc} % Encoding
\usepackage[utf8]{inputenc} % Encoding
\usepackage{lmodern} % Font
\usepackage{graphicx} % To insert figures
\usepackage{hyperref} % Clickable links
\usepackage{amssymb} % math stuffs
\usepackage{amsmath}
\hypersetup{
    colorlinks,
    citecolor=black,
    filecolor=black,
    linkcolor=black,
    urlcolor=black,
    pageanchor=false
}

\usepackage[left=3cm, right=3cm, bottom=4cm, top=4cm]{geometry} % Borders

% \newtheorem{defn}{Definition}
% \newtheorem{coro}{Corollary}
% \newtheorem{conjec}{Conjecture}
% \newtheorem{example}{Example}
% \newtheorem{Lem}{Lemma}
% \newtheorem{fact}{Fact}
% \newtheorem{prop}{{\bf Proposition}}

\renewcommand{\vec}[1]{\mathbf{#1}}


% \newcommand{\ol}[1]{\overline{#1}}
% \newcommand{\mr}[1]{\mathrm{#1}}
% \newcommand{\tx}[1]{\textrm{#1}}
\newcommand{\tr}[1]{{#1}^\top}

% \newcommand{\Om}{\Omega}
% \newcommand{\UU}{\mathcal{U}}
\newcommand{\CC}{\mathcal{C}}
% \newcommand{\TT}{\mathcal{T}}
% \newcommand{\QQ}{\mathcal{Q}}
% \newcommand{\II}{\mathcal{I}}
% \newcommand{\FF}{\mathcal{F}}
% \newcommand{\RR}{\mathbb{R}}
% \newcommand{\XX}{\mathcal{X}}
% \newcommand{\HH}{\mathcal{H}}
% \newcommand{\DD}{\mathcal{D}}
% \newcommand{\BB}{\mathcal{B}}
% \newcommand{\SSS}{\mathcal{S}}
% \newcommand{\NN}{\mathcal{N}}
\newcommand{\LL}{\mathcal{L}}
% \newcommand{\PP}{\mathrm{Pr}}
% \newcommand{\ppi}{\vec \pi}
\newcommand{\pp}{\vec p}
% \newcommand{\qq}{\vec q}
% \newcommand{\dd}{\vec d}
% \newcommand{\cc}{\vec c}
% \newcommand{\CX}{\overline{X}}
% \newcommand{\ax}{\tilde x}
% \newcommand{\kk}{\vec k}
% \newcommand{\va}{\vec a}
% \newcommand{\hx}{\hat{\vec x}}
% \newcommand{\hy}{\widehat{\vec y}}
% \newcommand{\hu}{\widehat{\vec u}}
% \newcommand{\hW}{\widehat{W}}
% \newcommand{\hD}{\widehat{D}}
% \newcommand{\hL}{\widehat{L}}
% \newcommand{\hU}{\hat{U}}
% \newcommand{\hA}{\hat{A}}
% \newcommand{\hY}{\hat{Y}}
% \newcommand{\hV}{\hat{V}}
% \newcommand{\hS}{\hat{\Sigma}}
% \newcommand{\mx}{\overline{\vec x}}
% \newcommand{\hX}{\hat{X}}
% \newcommand{\yy}{\vec y}
\newcommand{\vo}{\vec 1}
% \newcommand{\vz}{\vec 0}
% \newcommand{\ww}{\vec w}
% \newcommand{\zz}{\vec z}
% \newcommand{\bb}{\beta}
% \newcommand{\s}{s}
% \newcommand{\ee}{\vec e}
\newcommand{\uu}{\vec u}
\newcommand{\ff}{\vec f}
% \newcommand{\hh}{\vec h}
% \newcommand{\rr}{\vec r}
\newcommand{\aaa}{\boldsymbol{\alpha}}
% \newcommand{\bbb}{\boldsymbol{\beta}}
\newcommand{\ttt}{\boldsymbol{\theta}}
% \newcommand{\ssim}{\mathrm{sim}}
% \newcommand{\ssum}{\sum\limits}
% \DeclareMathOperator*{\argmax}{arg\,max}
% \DeclareMathOperator*{\argmin}{arg\,min}
% \DeclareMathOperator*{\trace}{tr}
% \DeclareMathOperator*{\diag}{diag}
% \DeclareMathOperator*{\rank}{rank}

% %\def\Rt{\mathbb{R}^{2}}
% %\def\Rn{\mathbb{R}^{n}}
\def\real{{\mathbb R}}
% %\def\natural{{\mathbb N}}
% %\def\C{{\mathbf \chi}}
% %\newcommand{\vecprod}[2]{\left\langle#1,#2\right\rangle}

% \def\httilde{\mbox{\tt\raisebox{-.5ex}{\symbol{126}}}}

% \newcommand{\vnu}{\boldsymbol \nu}
% \newcommand{\lu}{\tr\uu L\uu}
% \newcommand{\lz}{\tr\zz L\zz}
% \newcommand{\vF}{\vec F}
% \newcommand{\vv}{\vec v}

\begin{document}
    \begin{titlepage}
    \centering
    \includegraphics[width=0.3\textwidth]{figure/ets_logo}\par\vspace{1cm}
    {\scshape\LARGE ÉTS Montréal\par}
    \vspace{1cm}
    {\scshape\Large Class/goal\par}
    \vspace{1.5cm}
    {\huge\bfseries Main title\par}
    {\Large\itshape Sub title\par}
    \vspace{2cm}
    {\Large Hoel \textsc{Kervadec}\par}
    \vfill
    supervised by\par
    Dr.~John \textsc{Doe}

    \vfill

% Bottom of the page
    {\large \today\par}
\end{titlepage}

    \frontmatter
    \chapter*{\centering Abstract}
        Ce n'est qu'en essayant continuellement que l'on finit par réussir.
        En d'autres termes: plus ça rate et plus on a de chances que ça marche.
    \setcounter{tocdepth}{2}
    \tableofcontents

    \addcontentsline{toc}{chapter}{List of Figures}
    \listoffigures

    \addcontentsline{toc}{chapter}{List of Tables}
    \listoftables

    \setlength{\parskip}{5pt}

    \chapter{List of symbols}
    \begin{table}[h]
        \centering
        \begin{tabular}{|c|l|}
            \hline
                $I$ & Input image \\
            \hline
                $N$ & Number of pixels of $I$ \\
            \hline
                $\vec x = [x_1,x_2,...,x_N]$ & Set of random variables, written as a vector\\
            \hline
                $\{1,...,L\}$ & Different labels \\
            \hline
        \end{tabular}
    \end{table}

    \mainmatter
    \section{Introduction}
    This small template is aimed at making my life easier in the future.

    I will add with time a list of useful packages and what they do, to avoid too much bloat in new projects.

    \subsection{Figures}
        Below a few examples of figures, especially in a two columns environment.
        First, remember that Figure \ref{fig:intro:shadoko} is always here to help.

        \begin{figure}[h]
            \centering
            \includegraphics[width=0.8\columnwidth]{figure/intro/shadoko}
            \label{fig:intro:shadoko}
            \caption{He is watching over us.\cite{denis2012de}}
        \end{figure}


        For figures that takes the whole page, please see the code of Figure \ref{fig:intro:widsom}.
        \begin{figure*}[h]
            \centering
            \includegraphics[width=0.9\textwidth]{figure/intro/wisdom}
            \label{fig:intro:widsom}
            \caption{A few words of wisdoms from the Shadoks.}
        \end{figure*}

    \chapter{Content}
    \section{Figures}
        Below a few examples of figures, especially in a two columns environment.
        First, remember that Figure \ref{fig:intro:shadoko} is always here to help.

        \begin{figure}[h]
            \centering
            \includegraphics[width=0.4\textwidth]{figure/intro/shadoko}
            \label{fig:intro:shadoko}
            \caption{Professor Shadoko explains something.\cite{denis2012de}}
        \end{figure}

        \subsection{}
            For figures that takes the whole page, see the code of Figure \ref{fig:intro:widsom}.
            \begin{figure*}[h]
                \centering
                \includegraphics[width=1\textwidth]{figure/intro/wisdom}
                \label{fig:intro:widsom}
                \caption{Some French common wisdom.}
            \end{figure*}

    \section{Tables}
        Tables are great to display too much number, as in Table \ref{tab:supertable}

        \begin{table}[h]
            \centering
            \begin{tabular}{|r|r|}
                \hline
                X & X \\
                \hline
                .5 & .5 \\
                \hline
                1 & 1 \\
                \hline
            \end{tabular}
            \label{tab:supertable}
            \caption{X value over X}
        \end{table}

    \section{Maths}
        There is different ways to include maths, as $f(x) = \sum_i x_i$, \[ f(x) = \sum_i x_i, \] or even with a proper equation (Equation \eqref{eq:proper}):
        \begin{equation}
            \label{eq:proper}
            f(x) = \sum_i x_i
        \end{equation}

    \section{Conclusion}
    This was really useful.

    \appendix
    \include{input/appendix1}

    \bibliographystyle{plain}
    \bibliography{input/biblio}
\end{document}
