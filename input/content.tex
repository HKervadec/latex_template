\chapter{Content}
    \section{Figures}
        Below a few examples of figures, especially in a two columns environment.
        First, remember that Figure \ref{fig:intro:shadoko} is always here to help.

        \begin{figure}[h]
            \centering
            \includegraphics[width=0.4\textwidth]{figure/intro/shadoko}
            \label{fig:intro:shadoko}
            \caption{Professor Shadoko explains something.\cite{denis2012de}}
        \end{figure}

        \subsection{}
            For figures that takes the whole page, see the code of Figure \ref{fig:intro:widsom}.
            \begin{figure*}[h]
                \centering
                \includegraphics[width=1\textwidth]{figure/intro/wisdom}
                \label{fig:intro:widsom}
                \caption{Some French common wisdom.}
            \end{figure*}

    \section{Tables}
        Tables are great to display too much number, as in Table \ref{tab:supertable}

        \begin{table}[h]
            \centering
            \begin{tabular}{|r|r|}
                \hline
                X & X \\
                \hline
                .5 & .5 \\
                \hline
                1 & 1 \\
                \hline
            \end{tabular}
            \label{tab:supertable}
            \caption{X value over X}
        \end{table}

    \section{Maths}
        There is different ways to include maths, as $f(x) = \sum_i x_i$, \[ f(x) = \sum_i x_i, \] or even with a proper equation (Equation \eqref{eq:proper}):
        \begin{equation}
            \label{eq:proper}
            f(x) = \sum_i x_i
        \end{equation}